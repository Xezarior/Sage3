\documentclass{article}
\usepackage[utf8]{inputenc}
\usepackage{emerald}
\usepackage[T1]{fontenc}
\usepackage{lmodern}
\usepackage{ngerman}
\usepackage{amsfonts}
\usepackage{amssymb}
\usepackage{amsmath}
\usepackage{mathtools}
\usepackage{color}
\usepackage{xcolor}
\usepackage[bookmarks]{hyperref}
\usepackage{geometry}
\usepackage{ulem}
\usepackage{tikz}

\begin{document}
\title{Agenda}
\date{\today}
\author{Rü\c stü}
\maketitle
\begin{enumerate}

\item Metaregeln (Metaregeln regeln das Regeln Regeln)
\begin{enumerate}
\item Eine Person stellt ein Objekt vor; dieser Prozess wird nicht unterbrochen. Die Person präsentiert ausschließlich Inhalte, die sich auf dieses Objekt beziehen.
\item Die einzigen Fragen, die während dieses Prozesses gestellt werden, sind Verständnisfragen bzgl. des Objekts und der Inhalte der Vorstellung.\\
Kritik, Vorschläge, alternative Lösungswege und weiterführende Fragestellungen werden immer am Ende der Vorstellung diskutiert bzw. entschieden.
\item Eine Person, die eine Frage stellt, darf nicht unterbrochen werden.\\
Jeder Teilnehmer soll deswegen auf die Trennung von Inhalten und Fragen achten: es ist verboten, diskutiernotwendige Inhalte und Entscheidungen zusammen mit einer Frage in den Raum zu stellen, um dadurch Diskussionen bzgl. dieser Inhalte und Entscheidungen zu verhindern.
\item Die Inhalte der Vorstellung und anschließenden Diskussion des Objekts sollen sich nur auf dieses beziehen. Diese Einschränkung soll einen gemeinsamen Konsens bzgl. des Objekts erzwingen und die Verschleppung und Zerstreuung wichtiger Entscheidung verhindern.
\item Namen und Zahlen sind egal!\\
Alle Zahlen (außer 0 und 1) können im Nachhinein geändert werden. Die Festlegung konkreter Zahlenwerte ist Aufgabenbereich des Balancing und soll nicht Gegenstand dieser Diskussionen sein.\\
Namen sind rein ästhetischer Natur, ferner kann eine Entität über mehrere legitime Namen verfügen. Insofern ist die Namensgebung nicht Inhalt des Regelwerks.
\item Entscheidungen werden im Konsens getroffen. Jeder Entscheidungsträger muss sicher sein die optimale Entscheidung getroffen zu haben.
\item Entscheidungen werden nicht in Stein gemeißelt. Jede Entscheidung kann widerrufen werden.\\Je später eine Entscheidung geändert wird, desto schwieriger wird es diese Änderung umzusetzen. Dies kann unter Umständen dazuführen, dass gewisse Änderungen erst in der nächsten Version des Regelwerks umgesetzt werden können.
\end{enumerate}

\item geschützter Bereich
\begin{enumerate}
\item Info: Name, Hintergrund, Aussehen, ...
\item Erfahrungspunkte, Level, Klasse

\item Grundwerte
\begin{enumerate}
\item Attribute
\item Fertigkeiten
\item Talente
\item (Regel-)Buffs, Auras, Events
\end{enumerate}

\item Liste aktiver Buffs
\end{enumerate}

\item Konstanter Bereich
\begin{enumerate}
\item Items
\item Gliedmaßen
\item Buffs, Auras, Events
\item Schäden
\item Körperliche Zustände (Chemische Vergiftung; Verstrahlung; Alkohol; Blutverlust; Schlaf; Hunger)
\item Ort, Zeit, Lastzeit
\item Krankheitszustand
\item Situationen
\end{enumerate}

\item Attribute
\begin{enumerate}
\item Stärke
\item Geschick
\item Konstitution
\item Wahrnehmung
\item Intelligenz
\end{enumerate}

\end{enumerate}
\end{document}